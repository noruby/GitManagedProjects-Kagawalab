\section{はじめに}
%相転移/急冷/超伝導/鉄鋼/強相関電子系/準定常状態の電流励起/電荷秩序と競合した超伝導相
%なぜ秩序相の研究に価値があるのか読者任せになっている
%なぜIrTeを研究するのか: 非定常的な超伝導体ー絶縁体転移が知られているから
%顕微鏡構築と抵抗測定
%この系はスピン- 軌道相互作用の効果が大きく、非従来型の超伝導が予想されている。
\subsection{超伝導現象}
超伝導はある臨界温度以下で電気抵抗がゼロになる現象で、1911年にオランダのオンネスにより発見された。多彩かつ応用上有利な性質を超伝導物質は示すため、現在まで様々な超伝導物質に関して活発な研究開発が進められてきた。特に1986年に銅酸化物高温超伝導体が発見されて以来、銅酸化物を含む強相関電子系と呼ばれる物質群は精力的な研究の対象となった。

\subsection{ドープによる超伝導発現}
銅酸化物超伝導体は、母物質に少量の不純物を添加・元素置換したとき超伝導が発現する。この超伝導は電子間の相対的に強い反発力により引き起こされるもので、電荷やスピンなどの秩序化と起源を同じくしている。これらの超伝導相と秩序相は起源を同じくするため、一方の理解は他方の理解にも繋がる。またこの二つの相の競合は避けられない。

\subsection{超伝導発現への非平衡過程からのアプローチ: レーザー光入射と急冷}
近年、ある銅酸化物の秩序相に赤外光パルスを入射すると、過渡的に、超伝導相への転移が起こることが示された\cite{Fausti}。
IrTe$_2$は低温で電荷が一次元的に配列する電荷秩序状態となる(電荷密度波)が、この物質にパラジウム(Pd)やプラチナ(Pt)を添加すると臨界温度3Kで超伝導を示す。\cite{}近年、薄膜のIrTe$_2$に現れる電荷秩序を電流パルス印加後に急冷し破壊すると、競合する超伝導状態が現れることが実験により示された\cite{SC_IrTe2}。

\subsection{研究の目的}
しかしこのような急冷後に現れる超伝導相に関しては、いまだ理解されていないことが多い。本研究の目的はIrTe$_2$の超伝導状態をパルス光を用いても誘起できることを示し、その超伝導状態が実現される過程の理解を深めることである。筆者は2018年前期の実験において、偏光顕微鏡を用いてIrTe$_2$バルク試料の電荷秩序相を観察して、その理解を深めることを目標にした。まず偏光顕微光学系の設計と構築を行った。そして光学クライオスタット内の試料の抵抗を測定しながら温度を変化させ、偏光顕微鏡によって試料を観察した。本報告書では実験によって得られた成果と今後の課題について報告する。

%スズに関して(構造相転移など)

