\documentclass[11pt,a4paper]{jsarticle}
\usepackage[dvipdfmx]{graphicx}
\usepackage{graphicx}
\usepackage[top=25truemm,bottom=25truemm,left=25truemm,right=25truemm]{geometry}
\usepackage{comment}


\title{セラミックス材料学 試験問題の解き直し}
\author{東大物理工学科4年 03-153012 平松信義}
\date{\today}
\begin{document}
\maketitle

\section*{問1}
\begin{eqnarray}
\alpha_M &=& 2 \sum_{n}  \frac{(-1)^n}{n}\nonumber\\ 
&=& 2 log(1+1)\nonumber\\
&=& 2log2
\end{eqnarray}

\section*{問2}
(a) Y$_{3+}$イオンがZr$^{4+}$サイトに置換した場合、
\begin{eqnarray}
Y_2O_3 &\to& 2Y'_{Zr} +V_O^{\cdot\cdot} +3O_O^x
\end{eqnarray}

(b) Y$_{3+}$イオンが格子間位置に入った時、
\begin{eqnarray}
2ZrO_2 &\to& 2Zr'_{Y} +O"_{int} +3O_O^x
\end{eqnarray}

\section*{問3}
金属と比較して、セラミックスは転位を移動するのに必要な応力(パイエルス応力)が大きく、塑性変形しにくい。
セラミックスの電子雲は方向性を持ち局在化しているため、原子配列のわずかな変化でも大きなパイエルスポテンシャルの変化を生じ、大きなパイエルス応力となる。またbccやfcc、hcpなど簡単な結晶構造をとる金属と比較して、セラミックスでは一般にすべり面間隔が小さく、対応するバーガーベクトルも大きいため、パイエルス力は大きくなる。これはパイエルス力が小さく比較的塑性変形が起こりやすい金属と際立って異なる。
%パイエルス力が大きく転移は移動しにくい。このため、しばしば転移の運動および塑性変形に必要なせん断強度が破壊強度よりも大きく、塑性変形が起こらない。

また電荷が粒界にトラップされ電気特性に影響を及ぼす点もセラミックスは金属と異なる。


\section*{問4}
すべり分解した部分転位と比較して、上昇分解したものの方が滑り運動しやすい。なぜなら部分転位の相互作用によってパイエルス応力が小さくなるからである。

\section*{問5}
対応粒界とは特別な方位関係で結晶粒が接したときにできる界面で、規則的な構造を持ち粒界のエネルギーが比較的低い。
多結晶材料中に対応粒界のような 低エネルギー粒界を高頻度に導入することでランダム粒界の頻度を減らすことができ、ランダム粒界のネットワークを不連続にすることで腐食などの粒界劣化現象を抑制できる\cite{konakawa}。したがって対応粒界を制御することは、セラミックス工学においても重要である。

\section*{問6}
\textcircled{\scriptsize 1}表面
\textcircled{\scriptsize 2}ネッキング形成
\textcircled{\scriptsize 3}体積
\textcircled{\scriptsize 4}空隙
\textcircled{\scriptsize 5}粒径
\textcircled{\scriptsize 6}多面体
\textcircled{\scriptsize 7}球状
\textcircled{\scriptsize 8}粒界移動
\textcircled{\scriptsize 9}気孔
\textcircled{\scriptsize 10}体積
\textcircled{\scriptsize 11}粒界上

\section*{問7}
焼結体密度を向上させるために加圧下で焼結を行う手法にはホットプレス(HP)と熱間静水圧プレスがある。
HPは型を用い一軸加圧しながら加熱する方法である。焼結助剤の量が常圧焼結法より少なくて済むため、高強度かつ高温強度の低下の少ない焼結体が容易に得られる一方で、型を用いた一軸加圧のため、形状の複雑さや寸法に制限がある。

HIPはガス圧を利用して等方圧を加える手法である。HPと比較して、処理体の形状が大きく変化せず、変わる場合も相似的に収縮するという利点がある。

\section*{問8}
亀裂の進展と亀裂進展抵抗性との関係をR-曲線と呼び、材料の強度や破壊仕事などを述べる際にも重要な役割を演じる。亀裂の進展を理解する上で多くの場合、亀裂先端(フロンタルプロセスゾーン)以外の弾性変形エネルギーが支配的な役割を果たすが、その効果を現象論的に取り入れることができる。

\section*{問9}
\textcircled{\scriptsize 1}積層コンデンサ: 
小型で高周波特性がよく、高い電気容量をもつコンデンサには工学的に非常に大きな需要がある。その実現のためには、積層化で電束が貫く有効面積を大きくすることと、誘電率が大きな材料を使うことが有効である。特に酸化チタンやチタン酸バリウムが高誘電率材料として広く用いられる。

\textcircled{\scriptsize 2}圧電セラミックス: 
圧電効果により電圧と機械的な歪みを変換できる。この効果は音響・超音波素子や、微小アクチュエータ(ピエゾ素子)、簡便な高電圧源に用いられる。例えばチタン酸ジルコン酸鉛などが用いられる。

\textcircled{\scriptsize 3}バリスター: 
バルスタは高電圧がかかると抵抗が小さくなる非線形な電子部品で、他の電子部品を高電圧から保護するためのバイパスとして用いられる。酸化亜鉛やチタン酸ストロンチウムなどが用いられる。酸化亜鉛またはチタン酸ストロンチウムの結晶粒界のトンネル特性を応用したものである。

\section*{問10}
靭性向上には結晶粒界のコントロールが有効であり、結晶粒の微細化や不純物添加、気孔の除去が主なアプローチである。結晶粒を小さくすると亀裂は粒界を進展するため進展経路が複雑になり、亀裂進展の際に大きなエネルギーを吸収する。制御された不純物をドープすると粒界に局在して非晶質を形成し、亀裂の進展を阻止できることがある。また温度履歴制御により気孔を除去することも靭性向上に重要な影響を及ぼす。(さらにマイクロクラックを)


\section*{問11}
液体は小さな冷却速度でゆっくりと冷やすと、融点で固体結晶となる。しかし大きな冷却速度で素早く冷やすと融点でも結晶とならず、粘度の比較的大きな過冷却液体となる。さらに冷やすとガラス転移点で原子の移動ができなくなり、過冷却液体の構造がそのまま凍結されたガラスとなる。

\section*{問12}
a) 酸化物ガラスの網目形成体は、単独でもガラスを作ることができる。網目形成体には例えばSiO$_2$やB$_2$O$_3$などがある。\\
b) 網目修飾体はガラスの作成を容易にする目的で用いられる。網目修飾体には例えばLi$_2$OやNa$_2$Oなどがある。\\
c)中間元素は単独でガラスを作らないが、網目修飾体が共存すればガラスを形成しうる。中間元素には例えばAl$_2$O$_3$やTiO$_2$、ZrO$_2$などがある。

\section*{問13}
金属以外の結晶では主にフォノンが熱伝導を担うが、ガラスにおいてもフォノンがエネルギーを伝達する役割を果たす。しかし結晶と異なりガラスのフォノンは乱れ(非周期性)によって大きなレートで散乱されるため、ガラスの熱伝導は結晶に比べ小さい。

\section*{問14}
フロート製法とはスズとガラスの間の比重と融点の差を利用し、平板ガラスを製造する手法である。約1600$\bf ^\circ C$まで加熱された溶融ガラスを溶融スズの上に流し込むとガラスはスズの上に浮かびながら広がる。そのガラスをゆっくりと流しながら徐々に冷却すると、融点の違いから溶融スズ上のガラスだけが固まる。この手法ではガラス表面の磨きを必要とせず、両面が平行な平面が得られる。

\section*{問15}
単位胞1つあたりの双極子モーメント$p$は、
\begin{eqnarray}
p &=& p_{Ti} + p_{Ba} + p_{O}\\
 &=& 4e \times 0.12 + 2e \times 0.061 + 2e \times 0.036.
\end{eqnarray}

自発分極$P$は、
\begin{eqnarray}
P &=& \frac{p}{ca^2}\\
 &=& 0.1679 ({\bf C}/m^2)
\end{eqnarray}


\begin{thebibliography}{9}
\bibitem{konakawa} 粉川博之, 入門講座 粒界工学, マテリア 52, 1 (2013)
\end{thebibliography}

\end{document}