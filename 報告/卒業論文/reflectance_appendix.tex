\section{異方性のある反射率}
\label{sec:reflectance}
\subsection{一般の場合の反射率の導出}
比誘電率テンソル$\epsilon$が与えられたときの複素反射率テンソル$r$を導く。導出は文献\cite{dielectric_tensor_triclinic}に従う。

まず結晶の内部をz方向に伝搬する電磁波に関して考える。
結晶中のMaxwell方程式は真空の透磁率$\mu_0$と真空の誘電率$\epsilon_0$、比誘電率テンソル$\epsilon$を用いて、以下のように書ける。だだし結晶は磁性を持たず、電流密度と電荷密度がいたるところ0であるとした。
\begin{eqnarray}
\label{H1}
\mu_0 \nabla \cdot {\bf H} &=& 0,\\
\label{E1}
\nabla \times {\bf E} &=& -\mu_0 \frac{\partial \bf H}{\partial t},\\
\label{E2}
\epsilon_0 \nabla \cdot  (\epsilon {\bf E}) &=& 0,\\
\label{H2}
\nabla \times {\bf H} &=& \epsilon_0 \epsilon  \frac{\partial \bf E}{ \partial t}.
\end{eqnarray}

式\ref{E1}と式\ref{H2}をフーリエ変換すると、
\begin{eqnarray}
\label{E3}
{\bf k} \times {\bf E} = \mu_0 \omega {\bf H},\\
\label{H3}
{\bf k} \times {\bf H} = -\epsilon_0 \omega  \epsilon  {\bf E}.
\end{eqnarray}

式\ref{E3}に式\ref{H3}を代入してベクトル解析の公式を用いると、$k_0=\omega \sqrt{\epsilon_0 \mu_0}$を自由空間の波数として、
\begin{eqnarray}
\label{kk}
k^2 {\bf E} - ( {\bf k} \cdot {\bf E} ) {\bf k}=  k_0^2  \epsilon {\bf E}.
\end{eqnarray}

電磁波はz方向に伝搬するので、$k_x=k_y=0, k_z=k$である。式\ref{kk}から以下が言える。
\begin{eqnarray}
\label{eE}
\left(
    \begin{array}{ccc}
      \epsilon_{xx}-p & \epsilon_{xy} & \epsilon_{xz} \\
      \epsilon_{yx} &  \epsilon_{yy} -p & \epsilon_{yz} \\
      \epsilon_{zx} & \epsilon_{zy} & \epsilon_{zz} \\
    \end{array}
\right)
\left( \begin{array}{c} E_x\\ E_y\\ E_z \\ \end{array} \right)
=\left( \begin{array}{c} 0\\ 0\\ 0 \\ \end{array} \right).
\end{eqnarray}
ただし$p=k^2/k_0^2$はz方向の(複素)屈折率の二乗である。この式が零ベクトルでない電場$E$に関して成り立つとき左辺の行列は非正則で行列式は0である。行列式を0に等置した$p$に関する二次方程式を解くと、$p$は以下の$p^+$と$p^-$の二つの値をとる。
\begin{eqnarray}
p^\pm = (\frac{k^\pm}{k_0})^2 =\frac{1}{2u}[v \pm \sqrt{v^2-4uw}].
\end{eqnarray}
ただし、$u,v,w$を以下のようにおいた。
\begin{eqnarray}
u &=& \epsilon_{zz},\\
v &=& (\epsilon_{xx}+\epsilon_{yy})\epsilon_{zz} -\epsilon_{xz}^2-\epsilon_{yx},\\
w &=& \epsilon_{xx}\epsilon_{yy}\epsilon_{zz}-\epsilon_{xx}\epsilon_{yx}^2-\epsilon_{yy}\epsilon_{zx}^2 -\epsilon_{zz}\epsilon_{xy}^2 +2\epsilon_{xy}\epsilon_{yz}\epsilon_{zx}.
\end{eqnarray}

二つの$p^\pm$に対応して式\ref{eE}を満たす電場$E$は二つ存在する。すなわち結晶中をz方向に伝搬する電磁波には二つの異なるモード$E^+$と$E^-$が存在する。$K^\pm$を定数として、
\begin{eqnarray}
E_x^\pm(z) &=& K^\pm [\epsilon_{xy} \epsilon_{yz}- \epsilon_{xz}(\epsilon_{yy}-p^\pm)]exp(-ik^\pm_zz),\\
E_y^\pm(z) &=& K^\pm [\epsilon_{yx} \epsilon_{xz}- \epsilon_{yz}(\epsilon_{xx}-p^\pm)exp(-ik^\pm_zz)],\\
E_z^\pm(z) &=& K^\pm [(\epsilon_{xx}-p^\pm) (\epsilon_{yy}-p^\pm) -\epsilon_{xy}^2]exp(-ik^\pm_zz).
\end{eqnarray}

$E^\pm$を用いて、結晶中の電場$E^{crystal}$と磁場の強さ$H^{crystal}$は以下のように書ける。
\begin{eqnarray}
{\bf E}^{crystal}(z) &=& {\bf E^+(z) + E^-(z) } ,\\
{\bf H}^{crystal}(z) &=& \frac{1}{\mu_0 \omega} {\bf k} \times [{\bf E^+(z) + E^-(z)}].
\end{eqnarray}


以上の結果を用いて光の透過と反射の問題を考える。特に垂直入射の条件を考え、入射光の方向をz軸正の方向にとり結晶表面をxy面にとる。結晶表面に垂直入射した光の波面は表面に平行であるから、反射光と透過光の波面も表面に平行である。すなわち反射光はz軸の負の向き、透過光はz方向の正の向きに伝搬する。

入射光の電場をx方向にとり、その振幅を$E_0$とする。すなわち電場ベクトルは$(E_0,0,0)$である。このとき反射光の電場ベクトルは、複素反射率テンソル$r_{ij}$を用いて、$(r_{xx}E_0, r_{xy}E_0,0)$と書ける。
これから自由空間の電場$E^{free}$と磁場の強さ$H^{free}$は以下で表される。

\begin{eqnarray}
E_x^{free} (z) &=& E_0 exp(-ik_0 z) + r_{xx}E_0 exp(ik_0 z),\\
E_y^{free} (z) &=& r_{xy}E_0 exp(ik_0 z),\\
E_z^{free} (z) &=& 0,\\
{\bf H}^{free}(z) &=& \frac{1}{\mu_0 \omega} {\bf k} \times {\bf E^{free}(z)}.
\end{eqnarray}


結晶表面で電場$E$と磁場の強さ$H$の接線成分が連続であることを境界条件として仮定すると、代数計算から$r_{xx}$と$r_{xy}$に関して以下が言える。
\begin{eqnarray}
r_{xx} &=& \frac{2}{a^+ b^- - a^- b^+} (\frac{a^+ b^-}{1+\frac{k_1}{k_0}} - \frac{a^- b^+}{1+\frac{k_2}{k_0}} )-1,\\
r_{xy} &=& \frac{2b^+ b^-}{a^+ b^- - a^- b^+} (\frac{1}{1+\frac{k_1}{k_0}} - \frac{1}{1+\frac{k_2}{k_0}} ).
\end{eqnarray}

ただし$a^\pm$と$b^\pm$を以下のようにおいた。
\begin{eqnarray}
a^\pm &=& \epsilon_{xy} \epsilon_{yz} - \epsilon_{xz}(\epsilon_{yy}-p^\pm),\\
b^\pm &=& \epsilon_{yx} \epsilon_{xz} - \epsilon_{yz}(\epsilon_{xx}-p^\pm).
\end{eqnarray}

同様に入射光の電場をy方向にとって計算すると、以下が導かれる。
\begin{eqnarray}
r_{yy} &=& \frac{2}{b^+ a^- - b^- a^+} (\frac{b^+ a^-}{1+\frac{k_1}{k_0}} - \frac{b^- a^+}{1+\frac{k_2}{k_0}} )-1,\\
r_{yx} &=& \frac{2a^+ a^-}{b^+ a^- - b^- a^+} (\frac{1}{1+\frac{k_1}{k_0}} - \frac{1}{1+\frac{k_2}{k_0}} ).
\end{eqnarray}

時間反転対称性に関する議論から予想される通り、$r_{xy}=r_{yx}$が成り立つ。

\subsection{半導体スズの反射率}
\label{sec:reflectance_alpha}
半導体スズ(αスズ)のバルク試料は空間群$\rm m\overline{3}m$の対称性をもつことが、x線回折の結果から知られている \cite{THEWLIS}。また一般に磁性を持たない物質の誘電率テンソルは外部磁場が印加されていないとき、対称テンソルである\cite{landau}。

これらの半導体スズの対称性から誘電率テンソルは単位テンソルに比例し、以下のように1つの(複素)自由度$\epsilon_{xx}$を使って書ける。
\[
  \epsilon = \left(
    \begin{array}{ccc}
      \epsilon_{xx} & 0 & 0 \\
      0 & \epsilon_{xx} & 0\\
      0 & 0 & \epsilon_{xx} \\
    \end{array}
  \right).
\]
この誘電率テンソルを用いて反射率を計算すると、主軸の向きにかかわらず等方的な反射率($r_{xx}=r_{yy}, r_{xy}=r_{yx}=0$)が導かれる。すなわち半導体スズ表面に平行に入射した光は偏光状態を変えずに反射される。

\subsection{金属スズの反射率}
\label{sec:reflectance_beta}
金属スズ(βスズ)のバルク試料は空間群$4/mmm$の対称性をもつことが、x線回折の結果から知られている \cite{Wolcyrz}。また一般に磁性を持たない物質の誘電率テンソルは外部磁場が印加されていないとき、対称テンソルである\cite{landau}。

これらの金属スズの対称性から主軸がz方向を向いているとき、誘電率テンソルは以下のように二つの(複素)自由度$\epsilon_{xx}$と$\epsilon_{zz}$を使って書ける。
\[
  \epsilon = \left(
    \begin{array}{ccc}
      \epsilon_{xx} & 0 & 0 \\
      0 & \epsilon_{xx} & 0\\
      0 & 0 & \epsilon_{zz} \\
    \end{array}
  \right).
\]
このとき、この誘電率テンソルを用いて反射率を計算すると等方的な反射率($r_{xx}=r_{yy}, r_{xy}=r_{yx}=0$)が導かれ、c軸に平行に入射した光は偏光状態を変えずに反射される。

一方、主軸がz方向を向いていないとき誘電率テンソルは上に述べたものを回転変換したものであって、反射率も等方的でない($r_{xx}\neq r_{yy}, r_{xy}=r_{yx}\neq0$)。したがって、一般に金属スズ表面に平行に入射した光は反射するとき偏光状態を変える。


\subsection{直交偏光条件での信号強度}
\label{sec:diagonazed_reflectance}
反射率テンソルは複素対称行列${}^t r=r$であって、一般に直交行列またはユニタリー行列で対角化できない(と筆者は理解している)。しかし以下では簡単のため直交行列で対角化できるとする。このとき必要なら座標系を取り直すことで、XY平面に垂直入射する光の反射率テンソルを以下のように対角行列で表すことができる。
\[
  r = \left(
    \begin{array}{cc}
      r_1 &0  \\
      0 & r_2  \\
    \end{array} 
  \right) .
\]
ここで$r_1$と$r_2$は複素数である。 試料の光学軸をX軸に平行にとる。

今X軸に対して角$\alpha$回転した方向に偏光軸を持つ偏光子に光を通し、試料に入射する。さらに試料からの反射光を、偏光子と偏光軸が直交した検光子を通して検出することを考える。試料に入射する光のジョーンズベクトル$E_{inc}$は$E_{inc}=^t (E_0 cos\alpha, E_0 sin\alpha)$である。また検光子のジョーンズ行列$J$は以下で表される。
\[ J =\left(
    \begin{array}{cc}
      sin^2 \alpha & -cos\alpha sin\alpha  \\
      -cos\alpha sin\alpha & cos^2 \alpha  \\
    \end{array} 
  \right).
\] 

このとき、信号光の強度$I$は
\begin{eqnarray}
I(\alpha) &=& |J
 \left(
    \begin{array}{cc}
      r_1 &0  \\
      0 & r_2  \\
    \end{array} 
  \right)
  E_{inc}|^2\\
 &=& \frac{|E_0|^2}{8}|r_1-r_2|^2 \{1-cos(4\alpha)\},\\
\end{eqnarray}

したがって信号強度は、偏光子の偏光軸と試料の光学軸との相対角$\alpha$に対して、90度の周期性を持つ。すなわち$I(\alpha)=I(\alpha+\pi/2)$。この周期性は直交条件において、一般の反射率テンソルに関しても成り立つ。

\newpage