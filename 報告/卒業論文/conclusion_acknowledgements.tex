\section{結論}
実験に先立って筆者らは、スズリッチ領域で組成の異なるスズ-Ge合金をいくつか溶融・冷却し、その特性を評価した。その結果、溶融法と冷却速度が異なっても溶融したままの試料(As grwon)は金属スズ(超伝導スズ)であることを確かめた。その金属スズを家庭用冷蔵庫に保持すると、いくつかの試料は半導体スズに変換できた。さらにその中でも電気炉で1 mass \%Ge合金を溶融した試料(試料6)が、半導体スズと超伝導スズの変換を目指す本実験に適していることが分かった。

本実験では半導体スズの試料6に電流パルスを印加することでを金属スズへ変換できることを実証した。その金属スズは低温で半導体スズに再び戻らず、準安定であることを確かめた。また転移した金属スズが臨界温度\textcolor{red}{??K}で超伝導を示すことを示した。以上の成果は半導体スズと超伝導体スズの電流パルスによる不揮発変換を示したものである。

さらに筆者らは電流パルスを用いて同一試料に対して繰り返しα-β変換が可能であることを示した。さらに半導体スズ中に部分的に金属スズを書き込むことができることを示した。本実験で得られた知見は光パルスを用いてリソグラフィーを行う際、ただちに応用でき、非常に有益である。

\section*{謝辞}
賀川先生

大池さん

水野さん

海老野

\newpage