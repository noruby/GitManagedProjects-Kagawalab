\section{結論}
パルス印加実験に先立って、筆者らはスズにゲルマニウムを少量添加した合金を複数の組成・複数の方法で作成し、その特性を評価した。作成直後の試料はすべて金属スズ (超伝導スズ) であったが、家庭用冷蔵庫で低温に保持すると、いくつかの試料は相転移を起こした。筆者らはそれらの転移した試料が半導体的な抵抗-温度依存性を持ち、半導体スズの結晶構造をとることを X 線解析から確かめた。さらに、その中でも電気炉で 1 mass \%のGeを溶融した合金試料 (試料 6) が、相転移の再現性の良さから、半導体スズと超伝導スズの変換を目指す本研究の目的に適していることが分かった。

半導体スズの試料6に対して、電流パルスによる加熱・急冷を適用すると、半導体スズは金属スズへと変化した。パルス印加直後の冷却速度は$10^2$ K/s程度であり、パルスによって生成された金属スズは3.7K以下でゼロ抵抗を示した。以上の結果は、電流パルスによる半導体から超伝導体への不揮発変換に成功したことを意味している。このように、パルス加熱・急冷を用いた生成法がIrTe$_2$以外の物質系に適用できることを実証し、大きな抵抗変化を伴う超伝導生成を実現した。

さらに、パルス印加中に相転移が進行する様子を、光学顕微鏡を用いた実空間観測によって調べた。この測定により、半導体-超伝導変換に使用する電流パルスの強度を調節することで、半導体スズと金属スズの共存状態を生成できることを示した。共存状態の実現は、光を用いた局所加熱によって、半導体スズ中に金属スズを書き込むことが可能であることを示唆している。したがって、この知見は光リソグラフィによる超伝導回路の書き込みに応用することができる。

半導体スズと超伝導スズは、電気伝導度だけではなく、光学的な性質も異なる。半導体スズはエネルギー0.018eV (4.4THz)以下の光を透過するが、超伝導スズは反射する。これらの性質から、半導体中の超伝導回路のパターニングは低損失な電気回路のみならず、プラズモニック回路や量子コンピュータなどに有用であると考えられる。また、スズは地球上に豊富に存在する元素であり、低融点で人体への毒性も少ないため、取り扱いが容易である。このように、パルス加熱・急冷による半導体スズと超伝導スズの変換は、幅広い応用の可能性を秘めている。
\clearpage

\section*{謝辞}
賀川先生は研究の進め方や学科での発表などに関して、幅広く親切にご指導いただきました。また実験手法などに関して質問に伺うと、忙しいなか詳細に教えていただき大変勉強になりました。低温の四端子測定でフローブの電圧ケーブルを、様々なねじり方の中からどうねじって配線すればノイズが最も小さくなるかを教えて頂けた時は軽く感動したので特に印象に残っています。また進路の相談にのっていただくことも数多くありました。誠にありがとうございました。

大池さんは研究活動全般から実験の詳細に至るまで直接監督・ご指導いただきました。また研究が円滑に進むように様々な心づかいをいただきました。大変感謝致します。大池さんは学生の主体性とやる気を。
僕が卒論提出期限1週間前にインフルエンザA型に診断されたときにも、お気遣いと差し入れをいただきました。また僕の体調に配慮いただきながら、指示を出して頂けたのは大変助かりました。

理研賀川ユニットの佐藤さんには理研に行った時にお世話になりました。いつもにこやかに接してくださり、様々なお話ができたのは嬉しかったです。

理研賀川ユニットの松浦さんには理研に行った時にお世話になりました。僕の作業に付き添ってくださったり、物品の受け取りなどをしていただきました。

理研の吉川さんには理研でSn-Ge合金試料を作成した際に強力なサポートをいただきました。幅広い知識と経験から試料作成への的確なアドバイスのおかげで本実験を始めることができました。

理研の中島さんには理研で試料のX線解析を行なった際に指導いただきました。結晶構造の対称性をそらで計算しているのを見て、経験がある方に指導いただけるのは幸運なことだと思いました。

東大賀川研究室秘書の水野さんにはいつも暖かいお言葉とお菓子、お土産などをいただきました。また、水野さんのおかげもあって研究室の誕生日パーティーはいつも朗らかで面白かったです。

海老野くんは研究室の僕以外唯一の学生でした。
夜遅くまで実験したときや土日などに、一緒にお菓子を食べながら一緒に話したのはいい思い出です。彼からは多くの刺激を受けました。

最後に、集中できる環境を作り暖かくサポートしていただいた僕の両親と兄弟に深く感謝します。

\newpage