\section{結論}
まずパルス印加実験に先立って筆者らは、スズリッチ領域で組成の異なるスズ-Ge合金を溶融・冷却し、その特性を評価した。その結果、溶融したままの試料(As grwon)は金属スズ(超伝導スズ)であることを確かめた。その金属スズを家庭用冷蔵庫に保持すると、いくつかの試料は相転移を起こした。筆者らはそれらの転移した試料が半導体的な抵抗-温度依存性を持ち、半導体スズの結晶構造をとることをX線解析から確かめた。さらにその中でも電気炉で1 mass \%Ge合金を溶融した試料(試料6)が、半導体スズと超伝導スズの変換を目指す本実験に適していることが分かった。

本実験では半導体スズの試料6に電流パルスを印加することでを準安定な金属スズに変換できることを示した。その時の冷却レートは$10^2$K/sであり、先行研究\cite{oike}で超伝導が現れる最小の冷却レート$10^7$K/sの$10^{-5}$(10万分の1)程度だった。転移した金属スズは低温で半導体スズに再び戻らず、準安定であることを確かめた。また転移した金属スズが臨界温度\textcolor{red}{??K}で超伝導を示すことを示した。以上の成果はバルクのスズ試料において電流パルスを用いて半導体と超伝導体の不揮発変換を制御したものである。

さらに筆者らは電流パルスを用いて同一試料に対して繰り返しα-β変換が可能であることを示した。さらに半導体スズ中に部分的に金属スズを書き込むことができることを示した。本実験で得られた知見は光リソグラフィー技術などを用いて空間的な制御を繰り返し行う際、ただちに応用できる。
\clearpage

\section*{謝辞}
賀川先生は研究の進め方や学科での発表など、幅広く親切にご指導いただきました。また実験手法などに関して質問に伺うと、忙しいなか詳細に教えていただき大変勉強になりました。低温の四端子測定でフローブの電圧ケーブルをどう"ねじって"配線すればノイズが最も小さくなるかを教えて頂けた時は軽く感動したので特に印象に残っています。
また進路の相談にのっていただくことも数多くありました。誠にありがとうございました。

大池さんは研究活動全般から実験の詳細に至るまで直接監督・ご指導いただきました。また研究が円滑に進むように様々な心づかいをいただきました。大変感謝致します。大池さんはとにかく褒めて伸ばすのが上手な方だと思います。
僕が卒論提出期限1週間前にインフルエンザA型に診断されたときにも、お気遣いと差し入れをいただきました。また僕の体調に配慮いただきながら、指示を出して頂けたのは大変助かりました。

理研賀川ユニットの佐藤さんには理研に行った時にお世話になりました。いつもにこやかに接してくださり、様々なお話ができたのは嬉しかったです。

理研賀川ユニットの松浦さんには理研に行った時にお世話になりました。僕の作業に付き添ってくださったり、物品の受け取りなどをしていただきました。

理研の吉川さんには理研でSn-Ge合金試料を作成した際に強力なサポートをいただきました。幅広い知識と経験から試料作成への的確なアドバイスのおかげで本実験を始めることができました。

理研の中島さんには理研で試料のX線解析を行なった際に指導いただきました。結晶構造の対称性をそらで計算しているのを見ると、経験がある方に指導いただけるのは幸運なことだと思いました。

東大賀川研究室秘書の水野さんにはいつも暖かいお言葉とお菓子、お土産などをいただきました。また水野さんのおかげもあって研究室の誕生日パーティーはいつも朗らかで面白かったです。

海老野くんは研究室の僕以外唯一の学生でした。
夜遅くまで実験したときや土日などに、一緒にお菓子を食べながら一緒に話したのはいい思い出です。彼からは多くの刺激を受けました。

最後に、集中できる環境を作り暖かくサポートしていただいた僕の両親と兄弟に深く感謝します。

\newpage