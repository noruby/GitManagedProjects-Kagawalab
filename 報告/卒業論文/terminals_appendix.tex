
\section{サンプルの端子付け}
\label{sec:4terminal}
本実験ではサンプルの温度依存する抵抗率を四端子法から測定するために、IrTe$_2$のバルクサンプルとユニバーサル基板の間を金線を用いて電気的に接続した。金線とサンプル、金線とユニバーサル基板の間は銀ペーストを使って接着した。この章では端子付けの手順とその際に気をつけたことに関して、簡単にまとめる。

\subsection{端子づけに使った道具}

\begin{itemize}
\item 乾いた毛筆($250\mu m$程度)
\item 乾いた毛筆($500\mu m$程度)
\item 毛筆(銀ペースト用)
\item 毛筆(グリース用)\\
(*毛筆は用途によって印をつけておく(たとえば黒の塗りつぶし:銀ペースト;縞模様:細い)。毛は1mm程度の長さが良いと思う。長すぎると金線やサンプルを飛ばしてしまう。)
\item 銀ペーストを練るようじ
\item ピンセット
\item キムワイプ
\end{itemize}

材料・そのほかに便利な道具:
\begin{itemize}
\item 金線(太さ250um)
\item 銀ペースト
\item 光学顕微鏡
\item 金線を切るためのハサミ
\item 金線を置いておくゴムマット
\item 銀ペーストを練るスライドガラス
\item サンプル近くに置く銀ペーストのパレット
\item サンプルとパレットを乗せるスライドガラス
\item ユニバーサル基板(ピッチ2.5mm)
\item 両面テープ
\end{itemize}

毛筆の作り方:
\begin{enumerate}
\item 二液式接着剤を混ぜてつまようじの先に薄くつける
\item 一本の毛(腕毛、すね毛、まつげ、髪の毛など)を、接着剤に先を出して埋め込む
\item  接着剤が乾いたら、毛の長さをはさみで調整する
 \end{enumerate}

\subsection{作業を始める前に} 
\begin{itemize}
\item 前日はよく寝る
\item 机の上を整理整頓
\item 空調のスイッチを切る(作業終了後は再度スイッチを入れる)
\item 左右の接眼レンズの間隔とピントを調整する
\item 椅子の高さを調整
\end{itemize}

 
\subsection{四端子付けの手順} 
1. 準備
\begin{itemize}
\item ユニバーサル基板を適当な大きさにカットする
\item スライドガラスに両面テープでユニバーサル基板を貼り付ける。このとき銀ペーストを伸ばすパレットを横に作る。
\item 金線を必要なぶん切っておく(まっすぐで長さのそろった、汚れていないものが扱いやすい)
\item 銀ペーストをよく練って、サンプル横のパレットに適量のせる
\end{itemize}
 
2. サンプルの仮留め
\begin{itemize}
\item 基板にグリースを毛筆で少量つける。毛筆で伸ばす必要はない
\item 毛筆の先にグリースを微量つけてサンプルを持ち上げ、ユニバーサル基板につけたグリースの上に置く
\item サンプルを上から軽く押して、密着させる
\end{itemize}
 
3. 電流端子2本の取り付け
\begin{itemize}
\item まず金線一本をピンセットと毛筆で移動する。一方の端がサンプルの近くで、もう一方がユニバーサル基板の銅箔の近くにくるようにする
\item 金線と銅箔を銀ペーストでくっつける。銀ペーストが固まる前に、サンプル側の端がサンプルに触れるように微調整する
\item もう一本の金線に関しても同様に、銅箔と金線を銀ペーストでくっつけて固まるまで待つ
\item 銅箔側の銀ペーストが固まったら、サンプルと金線を銀ペーストでつなぐ。流れる電流がなるべく一様な密度になるように、横に広く接続することを意識する
 \end{itemize}
 
4. サンプルの持ち上げ
\begin{itemize}
\item サンプルと金線の間の銀ペーストが固まったら、金線と基板の間に毛筆を入れてサンプルを持ち上げる
 \end{itemize}
 
5. 電圧端子2本の取り付け
\begin{itemize}
\item 電流端子と同様に接続する。ただし、銀ペーストがサンプルに触れる面積が小さくなるように心がける。また測定中の低温でユニバーサール基板や金線が収縮して、サンプルと接合点に強い応力がかかるのを防ぐため、金線をユニバーサル基板の銅箔につけて固めた後少し曲げるとよいと思う。
 \end{itemize}
 
6. そのほか端末の取り付け
\begin{itemize}
\item PPMSのための端末をつくる
\end{itemize}

7. 最後に
\begin{itemize}
\item 記録のため写真をとっておく
\item 4端子間の抵抗を測定して3オーム程度であることを確認する
\end{itemize}
 

\subsection{作業が終わったら} 
\begin{itemize}
\item 空調の電源をオンにする
\item 顕微鏡の照明をオフにする
\item 机の上の片付け
\item サンプルをデシケータに入れる
\end{itemize}
 
 \subsection{作業のコツ} 
ピンセット
\begin{itemize}
\item 汚れたらキムワイプで拭いて先を綺麗に保つ
\item 金線はつよく掴まない、できるだけ平行につまむ
\item 先を保護するために、金線はゴムマットの上でつまむ
\end{itemize}
 
毛筆の使い方
\begin{itemize}
\item 銀ペーストを塗ったあとは毛を溶媒で洗って、キムワイプで拭く
\item つまようじを人差し指と中指で持つと疲れない
\item 小指の付け根を台につけて、左手を添えると震えにくい
\end{itemize}

銀ペーストの取り扱い
\begin{itemize}
\item 溶媒(コハク酸ジエチル)の液溜まりと固い銀ペーストの塊はスライドガラス上に離しておく。銀ペーストは少しずつ溶かして混ぜる。液溜まりでは銀ペーストを塗ったあとの毛筆を洗う。
\item パレットの銀ペーストは乾きやすいので、定期的に溶媒を足して練り直す
\item サンプルと金線を接続するときは、まず薄いペーストの表面張力でくっつける。乾くのを待って、濃いペーストで形を整える。はみ出した時は細く切った(2*20mm)キムワイプの先をちぎってねじったものに溶媒を含ませて拭く
\end{itemize}

その他
\begin{itemize}
\item 顕微鏡の倍率:高倍率で固定して、毛筆やピンセットはなるべく動かさない。スライドガラスとサンプルを動かすようにする
\item 休憩:一時間半に少なくとも一度休憩するのが望ましい
\end{itemize}

四端子測定が上手くいかない時
\begin{itemize}
\item 光学顕微鏡像の画像をとって、測定前と比べてみる
\item 各端子間の抵抗を測ってみる
\item 毛筆で端子を触ってみる
\end{itemize}

\newpage
