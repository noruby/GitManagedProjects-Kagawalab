\documentclass[a4paper,11ptj,twoside]{jsarticle}

\usepackage{lscape}
\usepackage[dvipdfmx]{graphicx}
\usepackage{here}
\usepackage{graphicx}
\usepackage{comment}
\usepackage{color}
\usepackage{afterpage}

\bibliographystyle{h-physrev3}

\setlength{\oddsidemargin}{\paperwidth}
  \addtolength{\oddsidemargin}{-\fullwidth}
  \setlength{\oddsidemargin}{.5\oddsidemargin}
  \iftombow
    \addtolength{\oddsidemargin}{-1in}
  \else
    \addtolength{\oddsidemargin}{-1truein}
  \fi
  \setlength{\evensidemargin}{\oddsidemargin}
  \if@mparswitch
    \addtolength{\evensidemargin}{\fullwidth}
    \addtolength{\evensidemargin}{-\textwidth}
  \fi
  \addtolength{\oddsidemargin}{0.4cm}
  \addtolength{\evensidemargin}{-0.4cm}

%\usepackage[top=25truemm,bottom=25truemm,left=30truemm,right=30truemm]{geometry}

%http://hamada.hatenablog.jp/entry/2018/09/09/183945
\usepackage{fancyhdr}

%\title{パルス加熱・急冷を用いた半導体スズと超伝導スズの不揮発変換}
%\author{東京大学工学部物理工学科 賀川研究室 平松信義}
%\date{\today}
\begin{document}
%\maketitle
%\bibliographystyle{prsty}

\pagestyle{empty}
\tableofcontents
\newpage

%タイトルの英小文字を大文字にする機能をOFFにする(http://miyahn.blog80.fc2.com/blog-entry-55.html)
\makeatletter
\def\ps@fancy{%
\def\sectionmark##1{\markboth{\ifnum \c@secnumdepth>\z@ \thesection\hskip 1em\relax \fi ##1}{}}%
\def\subsectionmark##1{\markright {\ifnum \c@secnumdepth >\@ne \thesubsection\hskip 1em\relax \fi ##1}}%
\ps@@fancy
\gdef\ps@fancy{\@fancyplainfalse\ps@@fancy}%
\ifdim\headwidth<0sp 
\global\advance\headwidth123456789sp\global\advance\headwidth\textwidth
\fi
}
\makeatother

\pagestyle{fancy}
\fancyhead{} % clear all header fields
\renewcommand{\chaptermark}[1]{\markboth{第\ \thechapter\ 章\,\, #1}{}}
\fancyhead[RO]{{\rightmark}}
\fancyhead[LE]{{\leftmark}}
%\renewcommand{\headrulewidth}{0pt} %ヘッダの罫線を消す 
\fancyfoot{} % clear all footer fields
\fancyfoot[LE,RO]{\thepage}%頁番号の表示を偶数頁と奇数頁に分けて指定.
%\fancyfoot[]{\thepage}
\cfoot{}

\setcounter{page}{1}
\input introduction/introduction.tex
\input samples/samples.tex
\input experiment/experiment.tex
\input results_discussions/results_discussions.tex
\input conclusion_acknowledgements.tex

\appendix
\input optics_microscopy/microscopy_appendix.tex
\input reflectance_appendix.tex
%\input terminals_appendix.tex
%\input review_Sn_film/review_Sn_film.tex

%\section{Optimization of the sputtering condition}
%\subsection{Substrate temperature and the density of Argon gas}
%Thornton 
%\subsection{Thickness of tin films}
%\subsection{InSb}
%\subsection{Voltage to }

\newpage

\bibliography{sotsuron.bib}

\end{document}
%コンパイルの仕方
%1. texファイルを一回コンパイル
%2. bibファイルを一回コンパイル
%3. texファイルを三回コンパイル