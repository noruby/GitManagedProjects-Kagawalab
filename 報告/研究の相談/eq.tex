\documentclass[11pt,a4paper]{jsarticle}
\usepackage[dvipdfmx]{graphicx}
\usepackage{graphicx}
\usepackage[top=20truemm,bottom=20truemm,left=25truemm,right=25truemm]{geometry}
\usepackage{comment}


%\title{電気回路上の量子電磁気学\\(ナノ科学 期末レポート)}
%\author{東大物理工学科4年 03-153012 平松信義}
%\date{\today}
\begin{document}
%\maketitle
%本レポートでは共振器の量子電磁気学 (cavity QED)に関して述べたあと、その応用である電気回路上の量子電磁気学(circuit QED)を説明する\cite{Schoelkopf}。

\begin{eqnarray}
γ &=& \sqrt{(R+j\omega L)(G+j\omega C)}\\
  &=& \alpha +j\beta
\end{eqnarray}
$R=0$のもとで、 伝搬長さ$L_{GHz}$と波長$\lambda_{ GHz}$は
\begin{eqnarray}
L _{GHz} &=&\frac{1}{2\alpha}\\
&=& \frac{1}{\sqrt{2\omega L(\sqrt{G^2+\omega^2 C^2} -\omega C)}}
\end{eqnarray}

\begin{eqnarray}
\lambda_{GHz} &=&\frac{2\pi}{\beta}\\
&=& \frac{2\pi}{\sqrt{\frac{\omega L}{2}(\sqrt{G^2+\omega^2 C^2} +\omega C)}}
\end{eqnarray}

\begin{eqnarray}
exp(-\frac{\lambda_{GHz}}{L _{GHz}}) &=& exp(-4\pi \frac{\sqrt{G^2+\omega^2 C^2} - \omega C}{G})\\
&=& exp(-4\pi \frac{G^3}{2C^3\omega^3}) +o[(\frac{G}{C\omega})^8]
\end{eqnarray}

\begin{thebibliography}{9}
\bibitem{Schoelkopf} R. J. Schoelkopf and S. M. Girvin, Wiring up quantum systems, Nature 451, 664 (2008)
\end{thebibliography}

\end{document}