\documentclass[aps,prb,reprint]{revtex4-1}

\usepackage[dvipdfmx]{graphicx}
\usepackage{graphicx}
\usepackage{bm}
\makeatletter

\begin{document}

\title{Study Plan}
\author{Nobuyoshi Hiramatsu}
\affiliation{Department of Applied Physics, the University of Tokyo.}
\maketitle

I am Nobuyoshi Hiramatsu, a senior student from the university of Tokyo. I would like to pursue my PhD degree in physics in Tsinghua university, and would find opportunities to work for facilitating scientific communications between China and Japan after finishing my PhD.

\section{Introduction of my background}
I was born in Toyama, the midwestern prefecture in Japan for the fourth child in my family.  In 2010, I entered Toyama national college of technology and majored electronics and controlling science. In 2015, I received my associate degree from the college, and transferred to the university of Tokyo to study applied physics. In the university, I had research opportunities and decided to put my efforts on condensed matter physics for my PhD. 
 
\section{Why do I choose to study in China}
China is getting to be a core to support scientific researches as China embraces great talents, huge communities and rapidly growing economy; and Japan has been an unique spot creating great researches in wide fields. However, only few Japanese are coming to China to study science today even though the connections between China and Japan has been significant for the development in science and technology. Therefore I decided to come to China for my PhD to find my unique contribution to facilitating the communications among the countries.

\section{Why do I choose to study at Tsinghua University}
Tsinghua is one of the top universities in China, and the faculty in physics department is internationally well-known for condensed matter physics.  Moreover, Tsinghua university and RIKEN in Japan maintain good relations for collaborating researches. I would like to assist this good relation directly for a Japanese PhD student at Tsinghua University in the near future, by making use of my experience at RIKEN. 

\section{Detailed study plan}
I would pursue my PhD degree, possibly in Prof Yu’s group. So far, I plan to work with magnetic oxide thin films and accomplish my PhD within averaged 60 months. The study will be arranged as bellow. 

\subsection{First Year (2019-2020)}
For the first year, I would put my efforts on study for catching up state-of-the-arts physics and for preparing the qualification exam. In particular I hope to study magnetisms and organic chemistry intensively to be my fundamental strength. Besides, I will attend the discussions in Prof Yu's group to follow the research projects ongoing. 

\subsection{Second Year (2020-2021)}
I would join the research group and do some preliminary research. I will then finish my graduate research proposal which can be related to the preliminary research, under the supervision of my research supervisor.  At the very beginning of my research in the group, I hope to focus on making myself be a convenient student and on contributing to my colleagues and supervisor, so as to maintain good relations for conducting independent research in the future. 

\subsection{Third to fifth year (2021-2024)}
At the latter part of my PhD at Tsinghua, I would focus on three things: accumulating knowledge, sticking to experiments, and making commitments for my group. I consider acquiring knowledge as well as publishing a paper are the most important parts for my PhD study to broaden my career options. I need to arrange myself to balance research and study. Doing hard works for the experiments is also essential, as well as being a good student.

\section{My plan after graduation}
After graduation, I plan to find a job for a scientist in China and Japan so as to connect the two society.  This work is very challenging and significant from my viewpoint, and I would be happy to have opportunities to work on the circumstance. Moreover, compared to the highly competitive and almost saturated job market in the U.S. and Japan, there is more various opportunities for young researcher in China in my opinion. 


%\vspace{-1em}
%\renewcommand{\refname}{}
\bibliography{personal_statement_tsinghua}

\end{document}