\documentclass[11pt,a4paper]{jsarticle}

\usepackage[dvipdfmx]{graphicx}
\usepackage{graphicx}
\usepackage[top=20truemm,bottom=20truemm,left=25truemm,right=25truemm]{geometry}
\usepackage{comment}

%https://qiita.com/Toru3/items/7ea1342da1e31f0c28c3
\usepackage{fancyhdr}
\usepackage{lastpage}
\fancypagestyle{mypagestyle}{%
\lhead{\thepage/\pageref{LastPage}}%ヘッダ左
\rhead{留学先での研究テーマ 平松信義}%ヘッダ右
\cfoot{}%\thepage/\pageref{LastPage}}%フッタ中央に"今のページ数/総ページ数"を設定
\renewcommand{\headrulewidth}{0.0pt}%ヘッダの線を消す
}
\pagestyle{mypagestyle}

\title{留学先での研究テーマ:\\準粒子からなるボーズ・アインシュタイン凝縮体の光制御}
\author{東京大学工学部物理工学科 平松信義}
\date{\today}
\begin{document}
\maketitle
\thispagestyle{mypagestyle}

\section{準粒子のボース・アインシュタイン凝縮}
%ボース・アインシュタイン凝縮 %関連分野(超伝導/超流動/ガス) 実験
ボース・アインシュタイン凝縮(BEC)は、ボーズ粒子が低温で協調して振る舞うことにより起こる巨視的な量子効果であり、1925年にアインシュタインが彼の論文の中で予言した。それ以来BECは超流動体中でのみ見つかっていたが、1995年に低温の希薄ガス中でも確認された\cite{Davis,Anderson}。理論的な予言に遅れて実験で観測されたのには理由がある。それは後述するようにレーザー技術が進歩し、極低温まで冷却する技術が発達したからである。観測が

%BCS理論により記述される金属の超伝導体や、近年盛んに研究されている銅酸化物高温超伝導体でも、複合粒子がBECを起こしている。さらに近年エネルギーと物理学の分野で長らく謎とされてきた、低温核融合反応に関しても、理論家がBECの観点から説明できることが示唆されている。

%半導体励起子
それ以降BECの実験研究は加速し、近年半導体中の準粒子である励起子がBECを起こすことが示された\cite{Yoshioka}。準粒子とは結晶中であるエネルギー量子であり、結晶など多体系の多彩な効果を説明するのに有用な概念である。
ここで半導体中の励起子は違いに相互作用するため、励起子のBECは温度1K 以下の超低温でのみ観測できることに注意する。

\section{超伝導}
超伝導とはある温度以下で電気抵抗がゼロになる現象で、1911年にオランダのオンネスが発見した。多彩かつ応用上有利な性質を超伝導物質は示すため、現在まで様々な超伝導物質に関して活発な研究開発が進められてきた。特に1986年に銅酸化物高温超伝導体が発見されて以来、特に銅酸化物を含む強相関電子系と呼ばれる物質群は精力的な研究の対象となった。

強相関電子系を特徴づける電子間の相対的に強い反発力は、超伝導を発現することがある一方で、電荷やスピンなどの秩序化も担う。これらの超伝導相と秩序相は起源を同じくするため、物質中で競合または共存\cite{Fausti,2D_SC}しており、一方の理解は他方の理解にも繋がる。また超伝導の前駆状態として秩序相が現れている可能性も示唆されている\cite{Valla1914}。したがって新しい超伝導物質や超伝導状態の開拓のためには秩序相の理解が有益である。

遷移金属カルゴゲナイドIrTe$_2$は低温で構造相転移し電荷が一次元的に配列する電荷秩序状態(電荷密度波)となるが、
この物質にパラジウム(Pd)を添加すると臨界温度3Kで超伝導を示す\cite{IrTe2Pd_SC}。この系は大きなスピン-軌道相互作用に起因して、非従来型の超伝導物質として注目されている。近年、電荷秩序状態にある薄膜のIrTe$_2$試料に電流パルスを印加すると、試料の急冷が起こり秩序相が破壊され、競合する超伝導状態が現れることが実験により示された\cite{SC_IrTe2}。この研究は急冷により電子の相を制御し超伝導を発現できることを示した点で画期的だと筆者は考える。しかし電流パルス印加では試料の冷却速度に上限がある。またこのような急冷後に現れる超伝導相に関しては、いまだ理解されていないことが多い。

\subsection{温度$T$と有効質量$m^*$}
すなわち準粒子のBECは量子多体系に典型的な現象である。
熱的ドブロイ長$\lambda_{TdB}$は以下で表される。
\begin{eqnarray}
\lambda_{TdB} &=& \frac{h}{\sqrt{2\pi m^* k_B T}}.\\
\end{eqnarray}

ここで励起子の有効執拗は大きく

%社会的意義
半導体中のは極めて新たしい分野であり、BECが実現された系は限られている。

%今後の発展

\section{近年の光技術の進展}
電磁波の存在は以来急速に進展し、いまだに盛んな研究が進められている。フォトンエネルギーと強度が小さい範囲では、非破壊であるという大きな利点もある。二つの例を挙げる。	

x線光源の図\ref{fig:brilliance}は、近年のx線光源の輝度のここ数十年の発展を示したものである\cite{Als-Nielsen}。指数関数的に
またコンパクトなx線レーザーの開発も進められている。例えばレーザー

さらに可視域と赤外域でもパルスレーザー光のますます向上している。

\section{BECを制御するために光技術を用いることの優位性}
そもそも、
MOT
超伝導体

\cite{Fausti}

\section{筆者のこれまでの経験}
筆者はこれまで一貫して光技術を

磁場下での励起子分光の実験の経験がある。と考える。

実験物理学の分野でしたい。


\section{結論}


\section{ボーズ・アインシュタイン凝集}
ボーズ・アインシュタイン凝集

\bibliography{SOP.bib}
\bibliographystyle{junsrt}


\end{document}
%コンパイルの仕方
%1. texファイルを一回コンパイル
%2. bibファイルを一回コンパイル
%3. texファイルを三回コンパイル